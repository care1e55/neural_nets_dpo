%!TEX TS-program = xelatex
\documentclass[12pt, a4paper, oneside]{article}

\usepackage{amsmath,amsfonts,amssymb,amsthm,mathtools}  % пакеты для математики

\usepackage[english, russian]{babel} % выбор языка для документа
\usepackage[utf8]{inputenc} % задание utf8 кодировки исходного tex файла
\usepackage[X2,T2A]{fontenc}        % кодировка

\usepackage{fontspec}         % пакет для подгрузки шрифтов
\setmainfont{Linux Libertine O}   % задаёт основной шрифт документа

\usepackage{unicode-math}     % пакет для установки математического шрифта
\setmathfont[math-style=upright]{Neo Euler} % шрифт для математики


% Размер страницы 
\usepackage[paper=a4paper, top=20mm, bottom=15mm,left=20mm,right=15mm]{geometry}
\usepackage{indentfirst}       % установка отступа в первом абзаце главы

\usepackage{setspace}
\setstretch{1.1}  % Межстрочный интервал
\setlength{\parskip}{4mm}   % Расстояние между абзацами
% Разные длины в латехе https://en.wikibooks.org/wiki/LaTeX/Lengths

\pagestyle{empty}

% Работа с картинками
\usepackage{graphicx}                  % Для вставки рисунков
\usepackage{graphics}
\graphicspath{{images/}{pictures/}}    % можно указать папки с картинками
\usepackage{wrapfig}                   % Обтекание рисунков и таблиц текстом
\usepackage{float}    

% Более-менее приятный синий, который не режет глаза
\usepackage{xcolor}
\definecolor{myblue}{rgb}{0.1, 0.45, 0.70}

% Немного подрифтуем списки и расстояния в них 
\usepackage{enumitem}
\newcommand*{\MyPoint}{\tikz \draw [baseline, fill=myblue,draw=blue] circle (2.5pt);}
\renewcommand{\labelitemi}{\MyPoint}

\setlist[itemize]{parsep=0.4em,itemsep=0em,topsep=0ex}
\setlist[enumerate]{parsep=0.4em,itemsep=0em,topsep=0ex}


% Работа с гиперссылками 
\usepackage{hyperref}
\hypersetup{
	unicode=true,           % позволяет использовать юникодные символы
	colorlinks=true,       	% true - цветные ссылки, false - ссылки в рамках
	urlcolor=blue,          % цвет ссылки на url
	linkcolor=red,          % внутренние ссылки
	citecolor=green,        % на библиографию
	pdfnewwindow=true,      % при щелчке в pdf на ссылку откроется новый pdf
	breaklinks              % если ссылка не умещается в одну строку, разбивать ли ее на две части?
}



% Счётчик для задачек 
\newcounter{problem}
\renewcommand{\theproblem}{\arabic{problem}}
\newcommand{\problemname}{Задача}

\newenvironment{problem}{
	\addtocounter{problem}{1}\noindent{ \color{myblue} \large\bfseries \problemname{} \theproblem \newline }
}{
	\par\bigskip
}

\newenvironment{solution}{
	{\bfseries Решение.}
}{
	\par\bigskip
}




% Немного своих команд 


\newcommand{\Sum}{\displaystyle\sum\limits}
\newcommand{\Max}{\max\limits}
\newcommand{\Min}{\min\limits}
\newcommand{\fromto}[3]{{#1}=\overline{{#2},\,{#3}}}
\newcommand{\floor}[1]{\left\lfloor{#1}\right\rfloor}
\newcommand{\ceil}[1]{\left\lceil{#1}\right\rceil}
\newcommand{\NN}{\mathbb{N}}
\newcommand{\RR}{\mathbb{R}}
\newcommand{\tild}{\widetilde}
\renewcommand{\hat}{\widehat}
\renewcommand{\emptyset}{\varnothing}
\renewcommand{\epsilon}{\varepsilon}
\newcommand{\ol}{\overline}

\newcommand{\mysetminus}{\mathbin{\fgebackslash}}



\usepackage{tikz, pgfplots}  % язык для рисования графики из latex'a

\usepackage{todonotes} % для вставки в документ заметок о том, что осталось сделать
% \todo{Здесь надо коэффициенты исправить}
% \missingfigure{Здесь будет Последний день Помпеи}
% \listoftodos --- печатает все поставленные \todo'шки


\begin{document}
	
	
\section*{Тятя! Тятя! Наши сети притащили мертвеца!}

\begin{problem}
	Маша услышала про машин лёрнинг и решила, что они и есть та самая Маша, которой этот лёрнинг принадлежит. Теперь она собрала два наблюдения: $x_1 = 1, x_2 = 2$, $y_1 = 2, y_2 = 3$ и собирается обучить линейную регрессию $y = \beta \cdot x$.  Она собирается сделать это тремя способами, и ей нужна ваша помощь!
	
	\begin{enumerate}
		\item  Получить теоретическую оценку методом наименьших квадратов.
		
		\item  Методом градиентного спуска. Она собирается в качестве скорости обучения взять $\eta = 0.1$.  В качестве стартовой точки она хочет использовать $\beta_0 = 0$. Обучение заканчивается после первого шага.  
		
		\item Методом стохастического градиентного спуска. Все параметры берутся такими же как в предыдущем пункте. Делается два шага. Сначала с первым наблюдением, потом со вторым.
	\end{enumerate}
\end{problem}


\begin{problem}
	Парни очень любят Олю, а Оля любит собирать персептроны и думать по вечерам о их весах и функциях активации. Сегодня она решила разобрать свои залежи из персептронов и как следует упорядочить их. 
	
	\begin{itemize}
		
		\item Для перцептрона 
		
		\begin{center}
			\definecolor{cqcqcq}{rgb}{0.7529411764705882,0.7529411764705882,0.7529411764705882}
			\begin{tikzpicture}[line cap=round,line join=round,x=1.0cm,y=1.0cm]
			\clip(-4,1) rectangle (3.306148366367316,4.5);
			\draw [line width=1.pt] (-3.,4.) circle (0.5cm);
			\draw [line width=1.pt] (-3.152000366859971,1.7166134938755313) circle (0.5cm);
			\draw [line width=1.pt] (-1.,3.)-- (-1.,2.);
			\draw [line width=1.pt] (-1.,2.)-- (1.5,2.);
			\draw [line width=1.pt] (1.5,2.)-- (1.5,3.);
			\draw [line width=1.pt] (1.5,3.)-- (-1.,3.);
			\draw [->,line width=1.pt] (-2.5,4.) -- (-1.0315953503775401,2.702352315488771);
			\draw [->,line width=1.pt] (-2.652000366859971,1.7166134938755313) -- (-1.,2.5);
			\draw [->,line width=1.pt] (1.5,2.5) -- (2.5,2.5);
			\draw (-3.4,2) node[anchor=north west] {$x_1$};
			\draw (-0.8,2.8) node[anchor=north west] {$\max(0,t)$};
			\draw (2.6,2.7) node[anchor=north west] {$y$};
			\draw (-1.8,3.9) node[anchor=north west] {$w_1$};
			\draw (-2.5,2.6) node[anchor=north west] {$w_2$};
			\draw (-3.2,4.25) node[anchor=north west] {$1$};
			\end{tikzpicture}
		\end{center}
		
		нужно подобрать веса так, чтобы он превращал $x_1 = 0$ в $y=1$, а $x_1 = 1$ в $y=0$.
		
		\item  Для перцепторона 
		
		\begin{center}
			\begin{tikzpicture}[line cap=round,line join=round,x=1.0cm,y=1.0cm]
			\clip(-4,0.5) rectangle (3.3,4.5);
			\draw [line width=1.pt] (-3.,4.) circle (0.5cm);
			\draw [line width=1.pt] (-3.,2.5) circle (0.5cm);
			\draw [line width=1.pt] (-3.,1.) circle (0.5cm);
			\draw [line width=1.pt] (-1.,3.)-- (-1.,2.);
			\draw [line width=1.pt] (-1.,2.)-- (1.5,2.);
			\draw [line width=1.pt] (1.5,2.)-- (1.5,3.);
			\draw [line width=1.pt] (1.5,3.)-- (-1.,3.);
			\draw [->,line width=1.pt] (-2.5,4.) -- (-1.0315953503775401,2.702352315488771);
			\draw [->,line width=1.pt] (-2.5,2.5) -- (-1.,2.5);
			\draw [->,line width=1.pt] (-2.5,1.) -- (-1.,2.2833448312560893);
			\draw [->,line width=1.pt] (1.5,2.5) -- (2.5,2.5);
			\draw (-3.3,4.25) node[anchor=north west] {$x_1$};
			\draw (-3.3,2.7) node[anchor=north west] {$x_2$};
			\draw (-3.3,1.26) node[anchor=north west] {$x_3$};
			\draw (-0.67, 2.8) node[anchor=north west] {$\max(0,t)$};
			\draw (2.6747622271028213,2.745742538897389) node[anchor=north west] {$y$};
			\draw (-2.1, 4) node[anchor=north west] {$w_1$};
			\draw (-2.25,3) node[anchor=north west] {$w_2$};
			\draw (-2.4,2.1) node[anchor=north west] {$w_3$};
			\end{tikzpicture}
		\end{center}
		
		Оля хочет по наблюдениям $x$ подобрать такие веса $w_i$, чтобы на выходе получились $y$. 
		
		\begin{center}
			\begin{tabular}{c|c|c|c}
				$x_1$ & $x_2$ & $x_3$ & $y$ \\
				\hline 
				$1$ & $1$ & $2$ & $0.5$\\
				\hline 
				$1$ & $-1$ & $1$ & $0$ \\
			\end{tabular}
		\end{center}
		
		
%		Марго хочет построить нейронную сеть так, чтобы она поделила плоскость на три части следущим образом: 
%		
%		\begin{center}
%			\begin{tikzpicture}[line cap=round,line join=round,x=1.0cm,y=1.0cm]
%			\clip(-3,-5) rectangle (5,-1);
%			\draw [line width=2.pt] (-3.,4.) circle (0.5cm);
%			\draw [line width=2.pt] (-3.152000366859971,1.7166134938755313) circle (0.5cm);
%			\draw [line width=2.pt] (-1.,3.)-- (-1.,2.);
%			\draw [line width=2.pt] (-1.,2.)-- (1.5,2.);
%			\draw [line width=2.pt] (1.5,2.)-- (1.5,3.);
%			\draw [line width=2.pt] (1.5,3.)-- (-1.,3.);
%			\draw [->,line width=2.pt] (-2.5,4.) -- (-1.0315953503775401,2.702352315488771);
%			\draw [->,line width=2.pt] (-2.652000366859971,1.7166134938755313) -- (-1.,2.5);
%			\draw [->,line width=2.pt] (1.5,2.5) -- (2.5,2.5);
%			\draw (-3.270567410142784,1.9563548477920916) node[anchor=north west] {$x_1$};
%			\draw (-0.17178835933248715,2.8036772444980356) node[anchor=north west] {$\max(0,t)$};
%			\draw (2.6727939724660277,2.7552588218291243) node[anchor=north west] {$y$};
%			\draw (-1.7695963074065464,3.8809871488813075) node[anchor=north west] {$w_1$};
%			\draw (-2.217466717093972,2.6342127651568465) node[anchor=north west] {$w_2$};
%			\draw (-3.088998325134368,4.2562299245653685) node[anchor=north west] {$1$};
%			\draw [->,line width=2.pt] (1.,-5.) -- (1.,-1.);
%			\draw [->,line width=2.pt] (-2.,-3.) -- (4.,-3.);
%			\draw [line width=2.pt,dash pattern=on 3pt off 3pt] (0.,-1.)-- (0.,-5.);
%			\draw [line width=2.pt,dash pattern=on 3pt off 3pt] (2.,-1.)-- (2.,-5.);
%			\draw (-1,-2) node[anchor=north west] {$1$};
%			\draw (2.5,-2) node[anchor=north west] {$1$};
%			\draw (1.25,-2) node[anchor=north west] {$0$};
%			\end{tikzpicture}
%		\end{center} 
		
		\item У Оли есть несколько вот таких перцептронов с неизвестной функцией активации (надо самому выбирать):
		
			\begin{center}
			\begin{tikzpicture}[line cap=round,line join=round,x=1.0cm,y=1.0cm]
			\clip(-4,0.5) rectangle (3,4.5);
			\draw [line width=1.pt] (-3.,4.) circle (0.5cm);
			\draw [line width=1.pt] (-3.,2.5) circle (0.5cm);
			\draw [line width=1.pt] (-3.,1.) circle (0.5cm);
			\draw [line width=1.pt] (-1.,3.)-- (-1.,2.);
			\draw [line width=1.pt] (-1.,2.)-- (1.5,2.);
			\draw [line width=1.pt] (1.5,2.)-- (1.5,3.);
			\draw [line width=1.pt] (1.5,3.)-- (-1.,3.);
			\draw [->,line width=1.pt] (-2.5,4.) -- (-1.0315953503775401,2.702352315488771);
			\draw [->,line width=1.pt] (-2.5,2.5) -- (-1.,2.5);
			\draw [->,line width=1.pt] (-2.5,1.) -- (-1.,2.2833448312560893);
			\draw [->,line width=1.pt] (1.5,2.5) -- (2.5,2.5);
			\draw (-3.3,4.25) node[anchor=north west] {$1$};
			\draw (-3.3,2.7) node[anchor=north west] {$x_1$};
			\draw (-3.3,1.26) node[anchor=north west] {$x_2$};
			\draw (-0.67, 2.8) node[anchor=north west] {$f(t)$};
			\draw (2.6747622271028213,2.745742538897389) node[anchor=north west] {$y$};
			\draw (-2.1, 4) node[anchor=north west] {$w_1$};
			\draw (-2.25,3) node[anchor=north west] {$w_2$};
			\draw (-2.4,2.1) node[anchor=north west] {$w_3$};
			\end{tikzpicture}
		\end{center}
	
	На плоскости проведены две прямые $x_1 + x_2 = 1$ и $x_1 - x_2 = 1$. 	
		
		\begin{center}
			\begin{tikzpicture}[line cap=round,line join=round,x=1.0cm,y=1.0cm]
			\clip(-2,-5) rectangle (4,-0.5);
			\draw [line width=2.pt] (-3.,4.) circle (0.5cm);
			\draw [line width=2.pt] (-3.152000366859971,1.7166134938755313) circle (0.5cm);
			\draw [line width=2.pt] (-1.,3.)-- (-1.,2.);
			\draw [line width=2.pt] (-1.,2.)-- (1.5,2.);
			\draw [line width=2.pt] (1.5,2.)-- (1.5,3.);
			\draw [line width=2.pt] (1.5,3.)-- (-1.,3.);
			\draw [->,line width=2.pt] (-2.5,4.) -- (-1.0315953503775401,2.702352315488771);
			\draw [->,line width=2.pt] (-2.652000366859971,1.7166134938755313) -- (-1.,2.5);
			\draw [->,line width=2.pt] (1.5,2.5) -- (2.5,2.5);
			\draw (-3.270567410142784,1.9563548477920916) node[anchor=north west] {$x_1$};
			\draw (-0.17178835933248715,2.8036772444980356) node[anchor=north west] {$\max(0,t)$};
			\draw (2.6727939724660277,2.7552588218291243) node[anchor=north west] {$y$};
			\draw (-1.7695963074065464,3.8809871488813075) node[anchor=north west] {$w_1$};
			\draw (-2.217466717093972,2.6342127651568465) node[anchor=north west] {$w_2$};
			\draw (-3.088998325134368,4.2562299245653685) node[anchor=north west] {$1$};
			\draw [->,line width=2.pt] (1.,-5.) -- (1.,-1.);
			\draw [->,line width=2.pt] (-2.,-3.) -- (4.,-3.);
			\draw (2.0191452664357303,-1.251365654023268) node[anchor=north west] {$1$};
			\draw (-0.3170436273392198,-2.4134077980771345) node[anchor=north west] {$0$};
			\draw [line width=2.pt,dash pattern=on 3pt off 3pt] (0.,-5.)-- (4.,-1.);
			\draw [line width=2.pt,dash pattern=on 3pt off 3pt] (0.,-1.)-- (4.,-5.);
			\draw (1.910203815430681,-4.0112157461512) node[anchor=north west] {$0$};
			\draw (3.2417104388257303,-2.4013031924099066) node[anchor=north west] {$0$};
			\end{tikzpicture}
		\end{center}
	
	Оле нужно собрать нейросетку, которая будет классифицировать объекты с плоскости так, как показано на картинке.
	\end{itemize}
\end{problem}



\begin{problem}

Попробуйте с помощью нейросеток с минимально возможным числом нейронов описать логические функции, заданные следущими таблицами истиности: 

\begin{center}
\begin{minipage}{0.3\linewidth} 
	\begin{tabular}{c|c|c}
	$x_1$ & $x_2$ & $x_1 \cap x_2$ \\
	\hline 
	$1$ & $1$ & $1$ \\
	\hline 
	$1$ & $0$ & $0$ \\
	\hline 
	$0$ & $1$ & $0$ \\
	\hline 
	$0$ & $0$ & $0$ \\
\end{tabular}
\end{minipage}
\hfill
\begin{minipage}{0.3\linewidth}
		\begin{tabular}{c|c|c}
		$x_1$ & $x_2$ & $x_1 \cup x_2$ \\
		\hline 
		$1$ & $1$ & $1$ \\
		\hline 
		$1$ & $0$ & $1$ \\
		\hline 
		$0$ & $1$ & $1$ \\
		\hline 
		$0$ & $0$ & $0$ \\
	\end{tabular}
\end{minipage}
\hfill
\begin{minipage}{0.3\linewidth}
		\begin{tabular}{c|c|c}
		$x_1$ & $x_2$ & $x_1 \mbox{ } XoR \mbox{ } x_2$ \\
		\hline 
		$1$ & $1$ & $0$ \\
		\hline 
		$1$ & $0$ & $1$ \\
		\hline 
		$0$ & $1$ & $1$ \\
		\hline 
		$0$ & $0$ & $0$ \\
	\end{tabular}
\end{minipage}
\end{center}

Первые два столбика идут на вход, третий получается на выходе.  Операция из третьей таблицы называется исключающим или. 

\end{problem}


\begin{problem}
Нарисуйте следущую функцию в виде нейросетки. 

\[
y = \max(0, 4 \cdot \max(0, 3x_1 + 4x_2 + 1) + 2 \cdot \max(0, 3x_1 + 2x_2 + 7) + 6)
\]


\end{problem}


\begin{problem}
	Дана нейросетка: 

\begin{center}
	\begin{tikzpicture}[line cap=round,line join=roundx=1.0cm,y=1.0cm]
	\clip(-3.68,0.12) rectangle (12.02,4.76);
	\draw [line width=2.pt] (-2.56,1.54) circle (0.5215361924162121cm);
	\draw [line width=2.pt] (-2.5,3.5) circle (0.5215361924162117cm);
	\draw [line width=2.pt] (0.,4.)-- (2.,4.);
	\draw [line width=2.pt] (2.,4.)-- (2.,3.);
	\draw [line width=2.pt] (2.,3.)-- (0.,3.);
	\draw [line width=2.pt] (0.,3.)-- (0.,4.);
	\draw [line width=2.pt] (0.,2.)-- (2.,2.);
	\draw [line width=2.pt] (2.,2.)-- (2.,1.);
	\draw [line width=2.pt] (2.,1.)-- (0.,1.);
	\draw [line width=2.pt] (0.,1.)-- (0.,2.);
	\draw [line width=2.pt] (4.,4.)-- (6.,4.);
	\draw [line width=2.pt] (6.,4.)-- (6.,3.);
	\draw [line width=2.pt] (6.,3.)-- (4.,3.);
	\draw [line width=2.pt] (4.,3.)-- (4.,4.);
	\draw [line width=2.pt] (4.,2.)-- (6.,2.);
	\draw [line width=2.pt] (6.,2.)-- (6.,1.);
	\draw [line width=2.pt] (6.,1.)-- (4.,1.);
	\draw [line width=2.pt] (4.,1.)-- (4.,2.);
	\draw [line width=2.pt] (8.,3.)-- (10.,3.);
	\draw [line width=2.pt] (10.,3.)-- (10.,2.);
	\draw [line width=2.pt] (10.,2.)-- (8.,2.);
	\draw [line width=2.pt] (8.,2.)-- (8.,3.);
	\draw [line width=2.pt] (-1.9787961021013818,3.4813855750750493)-- (0.,3.48);
	\draw [line width=2.pt] (-1.9787961021013818,3.4813855750750493)-- (0.,1.36);
	\draw [line width=2.pt] (-2.039699677075342,1.5041172191086443)-- (0.,1.36);
	\draw [line width=2.pt] (-2.039699677075342,1.5041172191086443)-- (0.,3.48);
	\draw [line width=2.pt] (2.,3.54)-- (4.,3.54);
	\draw [line width=2.pt] (2.,1.48)-- (4.,1.48);
	\draw [line width=2.pt] (2.,1.48)-- (4.,3.54);
	\draw [line width=2.pt] (2.,3.54)-- (4.,1.48);
	\draw [line width=2.pt] (6.,3.5)-- (8.,2.46);
	\draw [line width=2.pt] (6.,1.46)-- (8.,2.46);
	\draw (4.54, 3.75) node[anchor=north west] {$f(t)$};
	\draw (4.54,1.8) node[anchor=north west] {$f(t)$};
	\draw (0.52, 3.75) node[anchor=north west] {$f(t)$};
	\draw (0.54, 1.8) node[anchor=north west] {$f(t)$};
	\draw (8.6, 2.8) node[anchor=north west] {$f(t)$};
	\draw (-2.8, 1.8) node[anchor=north west] {$x_2$};
	\draw (-2.8, 3.75) node[anchor=north west] {$x_1$};
	\draw (6.78,3.8) node[anchor=north west] {$w_1^3$};
	\draw (6.76,2) node[anchor=north west] {$w_{2}^3$};
	\draw (-1.3,4.4) node[anchor=north west] {$w_{11}^1$};
	\draw (-1.5,2.2) node[anchor=north west] {$w_{21}^1$};
	\draw (-1.52,3.55) node[anchor=north west] {$w_{12}^1$};
	\draw (-1.58,1.4) node[anchor=north west] {$w_{22}^1$};
	\draw (2.76,4.4) node[anchor=north west] {$w_{11}^2$};
	\draw (2.68,1.4) node[anchor=north west] {$w_{22}^2$};
	\draw (2.52,2.25) node[anchor=north west] {$w_{21}^2$};
	\draw (2.48,3.55) node[anchor=north west] {$w_{12}^2$};
	\draw (10.5, 2.8) node[anchor=north west] {$y$};
	\end{tikzpicture}
\end{center}	
	
	\begin{enumerate}
		\item  Перепишите её как сложную функцию. 
		
		\item Запишите эту функцию в матричном виде.
		
		\item  Предположим, что $L(W_1,W_2,W_3) = \frac{1}{2} \cdot (y - \hat y)^2$ --- функция потерь, где $W_i$ --- веса $i-$го слоя.  Найдите производную функции $L$ по всем весам $W_i$. 
	\end{enumerate}
\end{problem}

\begin{problem}
	Изобразите для функции  $f(x,y) = x^2 + xy + (x + y)^2$ граф вычислений. Найдите производные всех выходов по всем входам. Опираясь на граф выпишите частные производные функции $f$.
\end{problem} 

\begin{problem}
	Функция $f(t) = \frac{e^t}{1 + e^t}$ называется сигмоидом. Покажите, что $f'(t) = f(t)(1 - f(t))$. 
\end{problem} 


\begin{problem}

Как-то раз Вовочка решал задачу классификации. С тех пор у него в кармане завалялась нейросеть: 

\begin{center}
\begin{tikzpicture}[scale = 1.5, line cap=round,line join=round,x=1.0cm,y=1.0cm]
\clip(-4.624679143882289,1.3686903642723092) rectangle (3.8521836267384844,4.8154607149701345);
\draw [line width=2.pt] (-2.5,2.5) circle (0.5cm);
\draw [line width=2.pt] (-0.5,2.5) circle (0.5cm);
\draw [line width=2.pt] (1.5,2.5) circle (0.5cm);
\draw [->,line width=2.pt] (-3.5,4.) -- (-2.831496141146421,2.874313115459547);
\draw [->,line width=2.pt] (-4.,2.5) -- (-3.,2.5);
\draw [->,line width=2.pt] (-2.,2.5) -- (-1.,2.5);
\draw [->,line width=2.pt] (0.,2.5) -- (1.,2.5);
\draw [->,line width=2.pt] (2.,2.5) -- (3.,2.5);
\draw [->,line width=2.pt] (-1.5,4.) -- (-0.8294354120380936,2.8761280490674572);
\draw [->,line width=2.pt] (0.5,4.) -- (1.1775032003746246,2.8820939861230355);
\draw (-3.686865302879703,4.4622580995276016) node[anchor=north west] {$1$};
\draw (-1.67726421501702,4.474437500060103) node[anchor=north west] {$1$};
\draw (0.2957986712481599,4.267387691007583) node[anchor=north west] {1};
\draw (-4.320194130569761,2.805859627107445) node[anchor=north west] {$x$};
\draw (3.1214195947884176,2.7693214255099416) node[anchor=north west] {$y$};
\draw (-3.7112241039447054,3.07380643882247) node[anchor=north west] {$w_1^1$};
\draw (-3.114433477852151,3.9750820782275555) node[anchor=north west] {$w_0^1$};
\draw (-1.7138024166145234,3.122524040952475) node[anchor=north west] {$w_1^2$};
\draw (-1.1535499921194723,4.13341428515007) node[anchor=north west] {$w_0^2$};
\draw (1.0387421037307276,4.109055484085068) node[anchor=north west] {$w_0^3$};
\draw (0.2836192707156588,3.0859858393549713) node[anchor=north west] {$w_1^3$};
\end{tikzpicture}
\end{center} 

В качестве функции активации используется сигмоид: $f(t) = \frac{e^t}{1 + e^t}$.  Есть два наблюдения: $x_1 = 1, x_2 = 5$, $y_1 =1$, $y_2 = 0$. Скорость обучения $\gamma = 1$. В качестве инициализации взяты нулевые веса. Как это обычно бывает, Вовочка обнаружил её в своих штанах после стирки и очень обрадовался.  Теперь он собирается сделать два шага стохастического градиентного спуска, используя алгоритм обратного распространения ошибки. Помогите ему. 
\end{problem}


\begin{problem}
	Дана нейросетка: 
	
	\begin{center}
		\begin{tikzpicture}[line cap=round,line join=roundx=1.0cm,y=1.0cm]
		\clip(-3.68,0.12) rectangle (12.02,4.76);
		\draw [line width=2.pt] (-2.56,1.54) circle (0.5215361924162121cm);
		\draw [line width=2.pt] (-2.5,3.5) circle (0.5215361924162117cm);
		\draw [line width=2.pt] (0.,4.)-- (2.,4.);
		\draw [line width=2.pt] (2.,4.)-- (2.,3.);
		\draw [line width=2.pt] (2.,3.)-- (0.,3.);
		\draw [line width=2.pt] (0.,3.)-- (0.,4.);
		\draw [line width=2.pt] (0.,2.)-- (2.,2.);
		\draw [line width=2.pt] (2.,2.)-- (2.,1.);
		\draw [line width=2.pt] (2.,1.)-- (0.,1.);
		\draw [line width=2.pt] (0.,1.)-- (0.,2.);
		\draw [line width=2.pt] (4.,4.)-- (6.,4.);
		\draw [line width=2.pt] (6.,4.)-- (6.,3.);
		\draw [line width=2.pt] (6.,3.)-- (4.,3.);
		\draw [line width=2.pt] (4.,3.)-- (4.,4.);
		\draw [line width=2.pt] (4.,2.)-- (6.,2.);
		\draw [line width=2.pt] (6.,2.)-- (6.,1.);
		\draw [line width=2.pt] (6.,1.)-- (4.,1.);
		\draw [line width=2.pt] (4.,1.)-- (4.,2.);
		\draw [line width=2.pt] (8.,3.)-- (10.,3.);
		\draw [line width=2.pt] (10.,3.)-- (10.,2.);
		\draw [line width=2.pt] (10.,2.)-- (8.,2.);
		\draw [line width=2.pt] (8.,2.)-- (8.,3.);
		\draw [line width=2.pt] (-1.9787961021013818,3.4813855750750493)-- (0.,3.48);
		\draw [line width=2.pt] (-1.9787961021013818,3.4813855750750493)-- (0.,1.36);
		\draw [line width=2.pt] (-2.039699677075342,1.5041172191086443)-- (0.,1.36);
		\draw [line width=2.pt] (-2.039699677075342,1.5041172191086443)-- (0.,3.48);
		\draw [line width=2.pt] (2.,3.54)-- (4.,3.54);
		\draw [line width=2.pt] (2.,1.48)-- (4.,1.48);
		\draw [line width=2.pt] (2.,1.48)-- (4.,3.54);
		\draw [line width=2.pt] (2.,3.54)-- (4.,1.48);
		\draw [line width=2.pt] (6.,3.5)-- (8.,2.46);
		\draw [line width=2.pt] (6.,1.46)-- (8.,2.46);
		\draw (4.54, 3.75) node[anchor=north west] {$f(t)$};
		\draw (4.54,1.8) node[anchor=north west] {$f(t)$};
		\draw (0.52, 3.75) node[anchor=north west] {$f(t)$};
		\draw (0.54, 1.8) node[anchor=north west] {$f(t)$};
		\draw (8.6, 2.8) node[anchor=north west] {$f(t)$};
		\draw (-2.8, 1.8) node[anchor=north west] {$x_2$};
		\draw (-2.8, 3.75) node[anchor=north west] {$x_1$};
		\draw (6.78,3.8) node[anchor=north west] {$w_1^3$};
		\draw (6.76,2) node[anchor=north west] {$w_{2}^3$};
		\draw (-1.3,4.4) node[anchor=north west] {$w_{11}^1$};
		\draw (-1.5,2.2) node[anchor=north west] {$w_{21}^1$};
		\draw (-1.52,3.55) node[anchor=north west] {$w_{12}^1$};
		\draw (-1.58,1.4) node[anchor=north west] {$w_{22}^1$};
		\draw (2.76,4.4) node[anchor=north west] {$w_{11}^2$};
		\draw (2.68,1.4) node[anchor=north west] {$w_{22}^2$};
		\draw (2.52,2.25) node[anchor=north west] {$w_{21}^2$};
		\draw (2.48,3.55) node[anchor=north west] {$w_{12}^2$};
		\draw (10.5, 2.8) node[anchor=north west] {$y$};
		\end{tikzpicture}
	\end{center}	
	
Для квадратичной функции ошибки $MSE(W_1, W_2, W_3) = (y - \hat y)^2$ выпишите все производные в том виде, в котором их было бы удобно использовать для алгоритма обратного распространения ошибки. 
\end{problem}


\begin{problem}
	Та, кому принадлежит машин лёрнинг собирается обучить нейронную сеть для решения задачи регрессии, На вход в ней идёт $12$ переменных, в сетке есть $3$ скрытых слоя. В пером слое $300$ нейронов, во втором $200$, в третьем $100$. 
	
	\begin{itemize}
		\item[a)] Сколько параметров предстоит оценить Маше?  Сколько наблюдений вы бы на её месте использовали? 
		\item[b)] Что Маша должна сделать с внешним слоем, если она собирается решать задачу классификации на два класса и получать на выходе вероятность принадлежности к первому классу? 
		\item[c)]  Что делать Маше, если она хочет решать задачу классификации на $K$ классов? 
	\end{itemize}
\end{problem}


\end{document}